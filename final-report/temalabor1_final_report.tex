\documentclass[lettersize,journal]{IEEEtran}
\usepackage{amsmath,amsfonts}
\usepackage{algorithmic}
\usepackage{algorithm}
\usepackage{array}
\usepackage[caption=false,font=normalsize,labelfont=sf,textfont=sf]{subfig}
\usepackage{textcomp}
\usepackage{stfloats}
\usepackage{url}
\usepackage{verbatim}
\usepackage{graphicx}
\usepackage{cite}
\hyphenation{op-tical net-works semi-conduc-tor IEEE-Xplore}
% updated with editorial comments 8/9/2021

\begin{document}

\title{Analyzing social networks in the European Parliament, and changes in the social network over time}

\author{{BERNÁT Ádám, MARITS Márton}
        % <-this % stops a space
\thanks{This paper was produced by deez nuts. They are in your mom.}% <-this % stops a space
\thanks{Manuscript received June 20, 2023; published two seconds later.}}

% The paper headers
\markboth{Journal of Literally Who Cares,~Vol.~1, No.~1, June~2023}%
{Shell \MakeLowercase{\textit{et al.}}: A Sample Article Using IEEEtran.cls for IEEE Journals}

\IEEEpubid{0000--0000/00\$00.00~\copyright~2021 IEEE}
% Remember, if you use this you must call \IEEEpubidadjcol in the second
% column for its text to clear the IEEEpubid mark.

\maketitle

\begin{abstract}
We did a couple things to some graphs which contained data about social networks in the European Parliament.
\end{abstract}

\begin{IEEEkeywords}
Social networks, European Parliament.
\end{IEEEkeywords}

\section{Introduction}

%THIS IS THE LAST SECTION WE WILL COMPLETE

In this section, we introduce what the hell we're even talking about, we cite a couple earlier papers which we definitely read and got inspired from, we talk about previous results and so on.

We then talk about how the paper itself is organized, and very lengthily talk about whan kinds of stuff is contained in each section. This is of course for the reader's convenience, and definitely not an attempt to make this paper longer than it should be.

\section{Our data}

Our data was acquired from blah blah blah...

\section{Preliminary analysis}

In this section, we talk about the conclusions we can draw from the data while analyzing it in its entirety.

\section{Changes to the social network over time}

In this section, we talk about how we divided the data based on time.

We divided our data based on time. Our goal was to make it possible to analyze the changes in the social environment based on time. We used major events that shaped European politics as breakpoints, because we expected that the social network might change drastically as a result of these events. The most important events we considered were the United Kingdom leaving the European Union (Brexit), on February 1st 2019, and the start of the Russian invasion of Ukraine on February 24th 2022. These events have undoubtedly shaped public opinion, and our research is centered around finding out whether they also influenced the social structure of the European Parliament.

\section{Conclusion}

The conclusion goes here.


\section*{Acknowledgments}

Thanks so much to everyone involved in creating this wonderful opportunity to uhhh, do things.

\begin{thebibliography}{1}
\bibliographystyle{IEEEtran}

\bibitem{asdf}
Asdf


\end{thebibliography}


\vfill

\end{document}


